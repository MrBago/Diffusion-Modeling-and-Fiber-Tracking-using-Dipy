\subsection{Local Tracking Class}
    The local tracking class implements the basic logic of the local fiber tracking algorithm. In the heart of the local tracking class is the main tracking loop. The main tracking loop uses a direction getter, a tissue classifier, and a seed to produce half a streamline. Because of antipodal symmetry, each seed is tracked in two opposing directions and the results joined to produce each streamline. Appendix \ref{appendix:algo} contains psudo-code for the local tracking algorithm including the basic loop. In dipy, the basic tracking loop is implemented using cython. Because this is the very inner loop of the tracking algorithm, it is important that this part of the code be optimized for performance. By implementing this loop in cython, we're able to compile this piece of the code and call the compiled code from python. Using cython has another major advantage. Because the abstract base classes \verb|DirectionGetter| and \verb|TissueClassifier| are both also defined in cython, their subclasses can implement the direction getter and tissue classifier interfaces either in cython or in python. While implementing these subclasses in python will be slower than using an optimized cython implementation, this design gives uses maximum flexibility. For development and testing of new concepts, developing in python allows users to quickly implement and try out different ideas. For many basic use cases, python is often fast enough, however for performance critical tasks users can implement direction getters and tissue classifiers in cython.
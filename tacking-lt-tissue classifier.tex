\subsection{Tissue Classifier}
    The tissue classifier is the component of the local tracking model which is aware of the different tissue types in the image. At it's most basic, it is the responsibility of the tissue classifier to terminate streamlines when they leave WM tissues. However, the tissue classifier can also be used to exclude streamlines which pass through tissues which are known not to contain axons. Currently dipy supports four point classes, \verb|TRACKPOINT|, \verb|ENDPOINT|, \verb|INVALIDPOINT|, and \verb|OUTSIDEIMAGE|. The choice of tissue classifier when tracking depends on the imaging and other data available for assessing tissue types. Based on all the available information about anatomical tissues, the tissue classifier must be able to determine which of the above point classes best represents each tracking point.
    
    Dipy currently has several standard tissue classifiers available. The threshold classifier, \verb|dipy.tracking.local.ThresholdTissueClassifier|, allows metric maps to be used, along with a threshold value, to classify tissues as either WM or not WM. This classifier is a good choice when the user does not have or want to use imaging data in addition to the diffusion MRI data set. In this case, simple metrics like FA or generalized fractional anisotropy (GFA) can identify WM tissues reasonably well. The binary classifier, \verb|dipy.tracking.local.BinaryTissueClassifier|, allows WM segmentations to be imported as binary masks. This option is a good choice when additional imaging data or better analysis tools are available for producing a white matter segmentation. For example freesurfer \cite{MIS} or FSL's FAST \ref{MIS} can be used to identify WM tissues from structural images, and those results can be used for fiber tracking. Lastly, the Anatomically-Constrained Tractography (ACT) classifier, \verb|dipy.tracking.local.ActTissueClassifier|, allows the use to provide two probabilistic tissue maps, an include map and an exclude map. This approach allows for a more fine grained classification of streamline points and allows some streamlines to be excluded completely based on the tissues traversed. This classifier is based on work by Smith et. al. \cite{Smith_2012} and (Girard?) \cite{MIS}.
    
    Dipy also allows user to define their own tissue classifiers. Here are the requirements for implementing a direction getter
\begin{enumerate}
\item The tissue classifier must inherit from \verb|dipy.tracking.local.tissue_classifier.TissueClassifier|.
\item The tissue classifier must implement a \verb|check_point| method.
\begin{enumerate}
\item The \verb|check_point| method takes one argument, \verb|point|.
\begin{enumerate}
\item \verb|point| is of type \verb|memoryview| with shape \verb|(3,)| and format \verb|'d'|.
\item \verb|point| is the current streamline position in voxel coordinates.
\end{enumerate}
\item The \verb|check_point| method should return a \verb|TissueClass|.
\begin{enumerate}
\item Currently the four valid tissue classes are:  \verb|TRACKPOINT|, \verb|ENDPOINT|, \verb|INVALIDPOINT|, and \verb|OUTSIDEIMAGE|.
\end{enumerate}
\end{enumerate}
\end{enumerate}
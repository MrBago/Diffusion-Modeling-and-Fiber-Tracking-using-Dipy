\section{Model the diffusion signal}
The first step in any diffusion MRI analysis, after pre-processing is complete, is to model the diffusion MRI signal. The model that one chooses depends on several factors. As with most modeling applications, the right model will balance several competing priorities including: simplicity of the model, assumptions built into the model, and accuracy of the result. In diffusion MRI, finding the optimal balance largely depends on the characteristics of the data to be modeled and the specifics of the application. For example, when the diffusion data to be modeled includes images acquired with different diffusion weightings, often called multiple q-shell data, a more complex model like Multitissue Constrained Deconvolution or Diffuison Spectrum Imaging can be used. If the data is acquired using a large number of diffusion gradient directions, more than 50, on a single q-shell with moderate to high b-value, then an intermediate model like QBall or Constrained Spherical Deconvolution might be the best choice. For data sets with a low b-value or a relatively few gradient directions, than a more simple model, such as DTI, is most appropriate. 

It's also important to consider the requirements of the application when picking a model. While a given data set might support more complex modeling, if the goal of the project is identify large, well defined white matter tracts in the brain, then the additional complexity and computational cost associated with a more complex model might not be justified. However, if the goal is to model areas of the cerebral white matter with finer structures and axonal crossings, than a simple model like DTI will simply fail to capture the underlying complexity of the biological architecture. The choice of model is often a subtle decision that can only be made empirically. Dipy aims to aid the user in the task of choosing the right models by providing a model API so that users can run the same tracking and analysis methods using different models in order to compare results.

\subsection{Model and Fit Classes}
Models in dipy serve two related functions. First a model can be used to fit diffusion MRI signals and estimate model parameters. Second, some dipy models provide a \verb|predict| method, which allows the model to predict diffusion MRI signals from model parameters. Here is a formal description of the model and fit interfaces in Dipy.

\begin{enumerate}
\item The model class.
\begin{enumerate}
\item Models classes are initialized with any required parameters to create model instances. In this context, a gradient table is considered a parameter and should be passed to the model in the constructor.
\item Models should have a \verb|fit| method. The \verb|fit| method should return a fit object, meaning an instance of a fit class.
\begin{enumerate}
\item The \verb|fit| method should take two parameters: data and mask. Data is an array of diffusion MRI signals and mask is a binary array where True values identify the non-background voxels of the image.
\end{enumerate}
\item The model may have a \verb|predict| method. The predict method should predict diffusion signals from model parameters.
\begin{enumerate}
\item The predict method should take two arguments: model parameters and a gradient table. The gradient table argument should be optional and the gradient table associated with the model should be used, by default.
\end{enumerate}
\end{enumerate}
\end{enumerate}
\begin{enumerate}
    \item The fit class.
    \begin{enumerate}
        \item Each fit object should have a \verb|model| attribute which will generally be the model object used to create the fit object.
        \item The fit class may also have a \verb|predict| method. The predict method of the fit class does not take model parameters as an argument, instead it estimates a signal prediction based on the model parameters associated with the fit instance.
        \item The fit object may have an \verb|odf| method which takes a \verb|Sphere| instance as an argument and returns a an the Orientation Distrubtion Function estimated by the model and sampled on the vertices of the \verb|Sphere| instance. The \verb|odf| method is used for peak finding and fiber tracking.
        \item The fit class may have attributes and methods specific to the model type which the fit represents. For example, a DTI fit has the attributes \verb|fa| and \verb|md| which are the fractional anisotropy and mean diffusivity, respectively, of the diffusion tensor.
    \end{enumerate}
\end{enumerate}

All the models in dipy follow this basic interface. Dipy only enforces this interface through "duck typing", meaning the interface is checked at run time on an as needed basis, as apposed to strictly enforced using a mechanism such as inheritance. The result of this design choice is that Dipy does not require all model classes to implement the entire model interface, however a model class that omits part of the interface may fail if it interacts with dipy functions or methods which require the missing portion of the interface.

\subsection{Dipy Untilties for Models}
Dipy provides some tools for making the implementation of new models easier. The first of these tools is the \verb|GradientTable| class. A gradient table object can be initialized from gradient table files, for example \verb|bvec| and \verb|bval| files comonly used in diffusion imaging, or simply an array of gradients by using the \verb|gradient_table| factory function in \verb|dipy.core.gradients|. A gradient table object allows a user to easily access different representations of gradient information. For example, the \verb|bvals| attribute of the gradient table object exposes the diffusion weighting, commonly known as the b-value, of each gradient. Similarly, the \verb|bvecs| attribute exposes the direction of each gradient as a unit vector and the \verb|gradients| attribute exposes the total gradient vector.

Dipy uses the \verb|Sphere| and \verb|HemiSphere| classes as discrete representations of the unit sphere. The \verb|HemiShere| class is a subclass of the \verb|Sphere| class, provides the same interface, and can be used anywhere Dipy requires a \verb|Sphere| instance.  Both \verb|Sphere| and \verb|HemiSphere| objects consist primarily of a set of dispersed points, or vertices, on the unit sphere. The \verb|Sphere| class also provides \verb|faces| and \verb|edges| attributes which provide neighbor information about the vertices of the sphere. Each face of the a \verb|Sphere| is triplet of integers between 0 and N - 1, where N is the number of points of the \verb|Sphere|. The faces, taken together, make up a mesh surface which approximates a sphere. Each edge of a \verb|Sphere| is a pair of integers between 0 and N - 1 and represents two neighbor points. Each point, has a neighbor relationship with the closest points to itself. This neighbor relationship is important for peak finding and fiber tracking applications.

Though the \verb|Sphere| and \verb|HemiSphere| classes are similar, there is one key difference. The \verb|HemiSphere| class incorporates an understanding of antipodal symmetry into it's definition of distance between points. On the surface of the sphere, we can define a distance function between two points, $\vec{v_1}$ and $\vec{v_2}$ as $d(\vec{v_1}, \vec{v_2}) = cos^{-1}(\vec{v_1} \cdot \vec{v_2})$. This is the intuitive definition of distance that is familiar to most people. Under this definition, the points $\vec{v}$ and $-\vec{v}$ are separated by a distance of \pi, the furthest possible distance between two points on a unit sphere. In the context of diffusion imaging, we want to be able to represent antipodally symmetric distance function where the points $v$ and $-v$ are in effect the same point. To represent this symmetry, we can define a new distance function on the sphere, $d(\vec{v_1}, \vec{v_2}) = cos^{-1}(|\vec{v_1} \cdot \vec{v_2}|)$. By using this new distance function, the \verb|HemiSphere| class is able to effectively represent an antipodally symmetric sphere with half as many points, without loosing important neighbor relations. Because the 

Dipy provides a \verb|default_sphere| object, which is an instance of \verb|HemiSphere|, in \verb|dipy.data|. The \verb|default_sphere| consists of 362 vertices dispersed on a unit hemisphere. We've found that this number of points provides a good balance between coverage and computational performance.

\Section{Model the diffusion signal}
The first step in any diffusion MRI analysis, after pre-processing is complete, is to model the diffusion MRI signal. The model that one chooses to use depends on several factors. As with most modeling applications, the right model will balance several competing priorities including: simplicity of the model, assumptions built into the model, and accuracy of the result. In diffusion MRI, finding the optimal balance largely depends on the characteristics of the available data and the specific of the application. For example, when the diffusion data to be modeled includes images acquired with different diffusion weightings, often called multi q-shell data, a more complex model like Multitissue Constrained Deconvolution or Diffuison Spectrum Imaging can be used. If the data is acquired using a large number of diffusion gradient directions, more than 50, on a single q-shell with moderate to high b-value, then an intermediate model like QBall or Constrained Spherical Deconvolution might be the best choice. For data sets with a low b-value or a relatively few gradient directions, than a more simple model, such as DTI, is most appropriate. It's also important to consider the requirements of the application when picking a model. While a given data set might support more complex modeling, if the goal of the project is identify large, well defined white matter tracts in the brain, then the additional complexity and computational cost associated with a more complex model might not be justified. However, if the goal is to model areas of the cerebral white matter with finer structures and axonal crossing, than a simple model like DTI will simply fail to capture the underlying complexity of the biological architecture. The choice of model is often a subtle decision that can only be made empirically. Dipy aims to aid the user in the task of choosing the right models by providing a model API so that users can run the same tracking and analysis methods using different models in order to compare results.
    
    
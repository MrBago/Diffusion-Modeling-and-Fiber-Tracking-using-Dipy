\section{Introduction}
    Fiber tracking, or tractography, is a developing set of techniques for identifying and measuring the properties of white matter (WM) pathways in mammalian brains. By measuring the restricted Brownian motion of water molecules in brain tissues, diffusion MRI gives us a way to examine the microstructural properties of those tissues. Diffusion models such as diffusion tensor imaging (DTI) or constrained spherical deconvolution (CSD) can be applied to diffusion MRI data sets to estimate tissue properties from diffusion weighted images. These tissue properties allow us to see into the microstructure of the axons that make up white matter tissues. These axons, and their surrounding myelin, restrict water diffusion therefore diffusion weighted MRI is sensitive to their diameter, density, integrity, and directionality. Fiber tracking allows us to take this analysis to the next level and begin to estimate the properties of individual WM tracts or the collections of tracts that make up the structural network of the brain. Dipy provides a unified tool set which allows researchers and developers to both analyze diffusion MRI data sets and develop new modeling and analysis methods. This chapter describes the design and organization of dipy's diffusion modeling and fiber tracking tools and how these tools can be extended to implement to analysis methods.
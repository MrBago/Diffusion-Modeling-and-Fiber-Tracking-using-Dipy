\section{Introduction}
    Fiber tracking, or tractography, is a developing set of techniques for identifying and measuring the properties of white matter (WM) pathways in mammalian brains. By measuring the restricted Brownian motion of water molecules in brain tissues, diffusion MRI gives us a way to examine the microstructural properties of those tissues. Diffusion models such as diffusion tensor imaging (DTI) or constrained spherical deconvolution (CSD) can be applied to diffusion MRI data sets to estimate tissue properties from diffusion weighted images. Fiber tracking allows us to take this analysis to the next level and begin to estimate the properties of individual WM tracts or the collections of tracts that make up the structural network of the brain.
    
    The structural network of the brain, or the structural connectivity of the brain, plays an important, but poorly understood, roll in aging and neurological disease.  

\textit{Oh, an empty article!}

You can get started by \textbf{double clicking} this text block and begin editing. You can also click the \textbf{Text} button below to add new block elements. Or you can \textbf{drag and drop an image} right onto this text. Happy writing!
